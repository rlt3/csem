\documentclass[12pt]{article}
\usepackage[english]{babel}
\usepackage[utf8x]{inputenc}
\usepackage{amsmath}
\usepackage{tikz}

\author{Raymon Leroy Todd, III}
\title{CS461 -- Project 4 Report}

\begin{document}
	\pagenumbering{gobble}
	\maketitle
	\pagebreak
	\pagenumbering{arabic}
	\section{csem}

	\subsection{The Problem}
    Using an existing Yacc grammar which parses a subset of the C programming
    language, construct a program which emits LLVM Intermediate Representation
    Code (IR) that can then be input to Clang to produce a binary executable.

	\subsection{My Design Approach}
    With the limited time for the project and the project's scope, I chose to
    not engineer any formal design. Instead, I decided to implement
    data-structures globally, as needed.  This approach does not produce a
    maintainable code base in the long-term but does produce a working code
    base in the short-term.

    \subsection{My Debugging and Testing Approach}
    Since the program should accept C programs, I constructed several small
    test programs which have an expected output value. These programs when
    compiled and executed produced a result that was tested against the
    expected value.  I automated this testing so that changes to the program
    could be verified as correct or, at least, not producing faulty programs.

    \subsection{Issues that arose}
    Two majors issues arose. First is that the given symbol table
    implementation in C when compiled with C++ compilers would segfault
    "randomly" due to its hash function implementation.
    Even after changing the hash function, the issue still persisted. So, I
    compiled all C code with gcc into objects and all C++ code with g++ into
    objects. Then I compiled all objects together with g++ into final
    executable.
    The second major issue was that parts of the code-base were in C and other
    parts were in C++. I needed some way to communicate complex LLVM objects
    between parts of the system. I resorted to using void-pointers to handle
    this discrepancy.  I personally don't mind the use of tactical
    void-pointers, but I still think of them as a "code-smell" that warrants
    being raised as an issue.

    \newpage
    \subsection{C Programs and their IR}
    The following are some test programs and IR they produce:

    \subsubsection{If Test}
    \paragraph{Program}
    \begin{verbatim}
int
main (int x) {
    if (x < 15)
        return 1;
    return 0;
}
    \end{verbatim}

    \paragraph{IR}
    \begin{verbatim}
; ModuleID = 'LEROY'
source_filename = "LEROY"

declare i32 @printf(...)

define i32 @main(i32 %x) {
entry:
  %x1 = alloca i32
  store i32 %x, i32* %x1
  %x2 = load i32, i32* %x1
  %slttmp = icmp slt i32 %x2, 15
  br i1 %slttmp, label %L0, label %L1

L0:                                               ; preds = %entry
  ret i32 1
  br label %L1

L1:                                               ; preds = %L0, %entry
  ret i32 0
}
    \end{verbatim}

    \newpage
    \subsubsection{Nested Backpatch Test}
    \paragraph{Program}
    \begin{verbatim}
int
main (int x) {
    int y;
    y = x - 1;
    if (x > 5 && x < 8 || x == 1)
        if (y && x || y == 0)
            return 100;
    return 0;
}
    \end{verbatim}

    \paragraph{IR}
    \begin{verbatim}
; ModuleID = 'LEROY'
source_filename = "LEROY"

declare i32 @printf(...)

define i32 @main(i32 %x) {
entry:
  %y = alloca i32
  %x1 = alloca i32
  store i32 %x, i32* %x1
  %x2 = load i32, i32* %x1
  %subtmp = sub i32 %x2, 1
  store i32 %subtmp, i32* %y
  %x3 = load i32, i32* %x1
  %sgttmp = icmp sgt i32 %x3, 5
  br i1 %sgttmp, label %L0, label %L1

L0:                                               ; preds = %entry
  %x4 = load i32, i32* %x1
  %slttmp = icmp slt i32 %x4, 8
  br i1 %slttmp, label %L2, label %L1

L1:                                               ; preds = %L0, %entry
  %x5 = load i32, i32* %x1
  %eqtmp = icmp eq i32 %x5, 1
  br i1 %eqtmp, label %L2, label %L7

L2:                                               ; preds = %L1, %L0
  %y6 = load i32, i32* %y
  %ccexpr = icmp ne i32 %y6, 0
  br i1 %ccexpr, label %L3, label %L4

L3:                                               ; preds = %L2
  %x7 = load i32, i32* %x1
  %ccexpr8 = icmp ne i32 %x7, 0
  br i1 %ccexpr8, label %L5, label %L4

L4:                                               ; preds = %L3, %L2
  %y9 = load i32, i32* %y
  %eqtmp10 = icmp eq i32 %y9, 0
  br i1 %eqtmp10, label %L5, label %L6

L5:                                               ; preds = %L4, %L3
  ret i32 100
  br label %L6

L6:                                               ; preds = %L5, %L4
  br label %L7

L7:                                               ; preds = %L6, %L1
  ret i32 0
}
    \end{verbatim}

    \newpage
    \subsubsection{Call Test}
    \paragraph{Program}
    \begin{verbatim}
int
foobar (int x)
{
    int ret;
    if (x < 15) {
        if (x == 1)
            ret = 5;
        else
            ret = 0;
    } else {
        ret = 10;
    }
    return ret;
}

int
main (int argc)
{
    return foobar(argc);
}
    \end{verbatim}

    \paragraph{IR}
    \begin{verbatim}
; ModuleID = 'LEROY'
source_filename = "LEROY"

declare i32 @printf(...)

define internal i32 @foobar(i32 %x) {
entry:
  %ret = alloca i32
  %x1 = alloca i32
  store i32 %x, i32* %x1
  %x2 = load i32, i32* %x1
  %slttmp = icmp slt i32 %x2, 15
  br i1 %slttmp, label %L0, label %L4

L0:                                               ; preds = %entry
  %x3 = load i32, i32* %x1
  %eqtmp = icmp eq i32 %x3, 1
  br i1 %eqtmp, label %L1, label %L2

L1:                                               ; preds = %L0
  store i32 5, i32* %ret
  br label %L3

L2:                                               ; preds = %L0
  store i32 0, i32* %ret
  br label %L3

L3:                                               ; preds = %L2, %L1
  br label %L5

L4:                                               ; preds = %entry
  store i32 10, i32* %ret
  br label %L5

L5:                                               ; preds = %L4, %L3
  %ret4 = load i32, i32* %ret
  ret i32 %ret4
}

define i32 @main(i32 %argc) {
entry:
  %argc1 = alloca i32
  store i32 %argc, i32* %argc1
  %argc2 = load i32, i32* %argc1
  %callret = call i32 @foobar(i32 %argc2)
  ret i32 %callret
}
    \end{verbatim}

    \newpage
    \subsubsection{For Loop Test}
    \paragraph{Program}
    \begin{verbatim}
int
main () {
    int num;
    int sum;
    int i;
    int j;
    num = 4;
    sum = 0;
    for (i = 0; i < num * num; i += 1)
        for (j = 0; j < num; j += 1)
            sum += 1;
    return sum;
}
    \end{verbatim}

    \paragraph{IR}
    \begin{verbatim}
; ModuleID = 'LEROY'
source_filename = "LEROY"

declare i32 @printf(...)

define i32 @main() {
entry:
  %j = alloca i32
  %i = alloca i32
  %sum = alloca i32
  %num = alloca i32
  store i32 4, i32* %num
  store i32 0, i32* %sum
  store i32 0, i32* %i
  br label %L0

L0:                                               ; preds = %L1, %entry
  %i1 = load i32, i32* %i
  %num2 = load i32, i32* %num
  %num3 = load i32, i32* %num
  %multmp = mul i32 %num2, %num3
  %slttmp = icmp slt i32 %i1, %multmp
  br i1 %slttmp, label %L2, label %L7

L1:                                               ; preds = <null operand!>, %L6
  %i4 = load i32, i32* %i
  %addtmp = add i32 %i4, 1
  store i32 %addtmp, i32* %i
  br label %L0

L2:                                               ; preds = %L0
  store i32 0, i32* %j
  br label %L3

L3:                                               ; preds = %L4, %L2
  %j5 = load i32, i32* %j
  %num6 = load i32, i32* %num
  %slttmp7 = icmp slt i32 %j5, %num6
  br i1 %slttmp7, label %L5, label %L6

L4:                                               ; preds = <null operand!>, %L5
  %j8 = load i32, i32* %j
  %addtmp9 = add i32 %j8, 1
  store i32 %addtmp9, i32* %j
  br label %L3

L5:                                               ; preds = %L3
  %sum10 = load i32, i32* %sum
  %addtmp11 = add i32 %sum10, 1
  store i32 %addtmp11, i32* %sum
  br label %L4

L6:                                               ; preds = <null operand!>, %L3
  br label %L1

L7:                                               ; preds = <null operand!>, %L0
  %sum12 = load i32, i32* %sum
  ret i32 %sum12
}
    \end{verbatim}

	\section*{}
\end{document}
